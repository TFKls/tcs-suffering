\begin{definition}
	$d$-wymiarową kostką o boku $m$ nazywamy zbiór $[m]^d = \set{(x_1,\ldots,x_d) : \forall_i \ x_i \in [m]}$.
\end{definition}

\begin{definition}
	$L\subset [m]^d$ nazywamy linią kombinatoryczną, jeśli istnieje niepusty zbiór $I\subset[d]$ oraz $\set{y_i}_{i\in[d]\setminus I}$ takie, że $L$ jest postaci $\set{(x_1^\alpha,\ldots,x_d^\alpha) : \alpha\in[m]}$, gdzie $$x_i^\alpha = \left\{ \begin{array}{lr}
			\alpha & i\in I             \\
			y_i    & i\in[d]\setminus I
		\end{array} \right.,$$
	czyli istnieje zbiór wymiarów, na których współrzędne rosną od $1$ do $m$, a na pozostałych są ustalone. Zbiór $I$ nazywamy aktywnym zbiorem.
\end{definition}

\begin{definition}[Alternatywna definicja]
	Dla niektórych bardziej intuicyjna może być równoważna definicja linii kombinatorycznej poprzez funkcje.
	Linią kombinatoryczną w $[m]^d$ nazywamy zbiór postaci $\set{(f_1(i), f_2(i), \ldots, f_n(i)) : i \in [m]}$,
	gdzie każda z funkcji $f_i : [m] \to [m]$ jest równa $I : [m] \ni x \mapsto x \in [m]$ lub $K_v :
		[m] \ni x \mapsto v \in [m]$ dla pewnego $v$, oraz istnieje indeks $j$ spełniający $f_j = I$.
	Zbiorem aktywnym jest wtedy zbiór $\mathcal{I} = \set{i \in [m]: f_i = I}$,
	a przez $x_i^{\alpha}$ oznaczamy $f_{\alpha}(i)$ (aby zachować notację z poprzedniej definicji).
\end{definition}

\begin{theorem}[Hales-Jewett]
	Dla dowolnych $m, k \in \natural_1$ istnieje $N \in \natural$ o tej własności, że
	dla każdego kolorowania $c: {[m]}^N \to [k]$ istnieje monochromatyczna linia kombinatoryczna.
	Najmniejszą liczbę $N$ spełniającą powyższe nazywamy $\text{HJ}(m, k)$.
\end{theorem}
\begin{proof}
	Dla linii kombinatorycznej $L$ niech $L^-$ i $L^+$ oznaczają jej pierwszy i ostatni punkt. Mówimy, że linie $L_1,\ldots, L_s$ zą zogniskowane w $f$, jeśli $L^+_i = f$ dla każdego $i\in[s]$.
	Dodatkowo mówimy, że są kolorowo zogniskowane, gdy wszystkie linie $L_i\setminus\set{f}$ są monochromatyczne i w innym kolorze.
	Przeprowadzimy dowód indukcyjny po $m$. Baza $m=1$ jest oczywista (istnieje tylko jeden punkt kostki).

	Zdefiniujmy $T = \text{FHJ}(k,s,m)$ jako najmniejszą taką liczbę, że każde $k$-kolorowanie $[m]^T$ albo zawiera monochromatyczną linię kombinatoryczną, albo zawiera $s$ kolorowo zogniskowanych linii.
	Zauważmy, że podstawienie $s=k$ daje nam naszą tezę, bo co najmniej jedna ze zogniskowanych linii ma wtedy taki sam kolor, jak punkt zogniskowania.
	Dowodzimy istnienie tej rodziny liczb indukując się po $s$. Dla $s=1$ wystarczy postawić $\text{FHJ}(k,1,m) = \text{HJ}(k,m-1)$ (istnieje krótsza monochromatyczna linia).
	Niech $n = \text{FHJ}(k,s-1,m)$ i $n' = \text{HJ}(k^{m^n},m-1)$. Pokażemy, że $\text{FHJ}(k,s,m) \le n+n'$.

	Weźmy dowolne $k$-kolorowanie $\phi$ kostki $[m]^{n+n'}$. Można dla niego zdefiniować $(k^{m^n})$-kolorowanie $\phi'$
	kostki $[m]^{n'}$ jako kolorowanie produktowe wszystkich punktów o ustalonych pierwszych $n'$ współrzędnych w $[m]^{n+n'}$
	(zatem każda kostka $[m]^n$ jest osobnym kolorem). Z definicji $n'$ w $\phi'$ istnieje linia $L$ w $[m]^{n'}$ o aktywnym
	zbiorze $I$ taka, że skrócona linia $L\setminus\set{L^+}$ jest monochromatyczna. To znaczy, że dla $a\in[m]^n$
	i $b,b'\in L\setminus\set{L^+}$ zachodzi $\phi((b,a)) = \phi((b',a))$. Można więc zdefiniować kolorowanie
	$\phi''$ kostki $[m]^n$ jako $\phi''(a) = \phi((b,a))$ dla dowolnego $b\in L\setminus\set{L^+}$.
	Monochromatyczna linia w $\phi''$ jest też monochromatyczną linią w $\phi$ (ten sam zbiór aktywny, w $n'$ pierwszych wymiarach same nieaktywne leżące na $L$).
	Załóżmy więc, że w $\phi''$ nie ma monochromatycznej linii. Z definicji $n$ mamy w $\phi''$ zbiór $s-1$ kolorowo zogniskowanych linii
	$L_1,\ldots,L_{s-1}$ o aktywnych zbiorach $I_1,\ldots,I_{s-1}$ i ognisku $f$. Dla każdego $i$ zdefiniujmy $L'_i$ jako linię w
	$[m]^{n+n'}$ o pierwszym punkcie $(L^-,L_i^-)$ i aktywnym zbiorze $I\cup I_i$, a $L'_s$ jako linię o pierwszym punkcie $(L^-,f)$
	i aktywnym zbiorze $I$. O ile w $\phi$ nie ma monochromatycznej linii, to właśnie zdefiniowaliśmy $s$ kolorowo zogniskowanych linii
	o ognisku $(L^+,f)$, bo jeśli $f$ był innego koloru niż wszystkie zogniskowane w nim linie, to $L'_s$ będzie miała całkiem nowy kolor.
	To kończy dowód.
\end{proof}

