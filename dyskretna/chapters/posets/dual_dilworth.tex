\begin{theorem}[Twierdzenie dualne do twierdzenia Dilwortha]
	Długość maksymalnego łańcucha w posecie wynosi
	Długość maksymalnego łańcucha w posecie jest równa ilości antyłańcuchów w najmniejszym pokryciu antyłańcuchowym
	(tj. rozkładzie posetu na podzbiory będące antyłańcuchami).
	Długość tą dla danego posetu nazywamy szerokością i oznaczamy $\posetheight(P)$.
\end{theorem}

\begin{proof}
	W jedną stronę nierówność jest trywialna -- mając łańcuch o długości $k$, każdy z jego
	elementów musi trafić do innego antyłańcucha. Dowodzimy więc nierówność w drugą stronę.
	Niech $(P, \leq)$ będzie posetem -- zdefiniujmy sobie funkcję $\varphi$ idącą z elementów
	$P$ w liczby naturalne, taką że $\varphi(x)$ jest to moc najdłuższego łańcucha w $P$,
	którego maksimum wynosi $x$. Zauważmy, że $\varphi$ może przyjmować jedynie wartości
	w zakresie $1 \dots k$, bo $k$ to długość najdłuższego łanćucha w ogóle. Zauważamy,
	że wszystkie elementy $P$ które przechodzą na jakąś liczbę $m$ muszą być ze sobą
	nieporównywalne, a więc formować antyłańcuch. Gdyby tak nie było i istniałyby jakieś
	elementy $x, y$, takie że, bez straty ogólności, $x \leq y$ i
	$\varphi(x) = \varphi(y) = z$, to łańcuch w którym $x$ jest elementem maksymalnym
	i który ma długość $z$ możemy ,,rozszerzyć'' dodając do niego $y$, które stałoby się
	nowym elementem maksymalnym; tym samym maksymalna długość łańcucha w którym $y$
	byłoby elementem maksymalnym wynosiłaby nie $z$, a $z+1$, co prowadziłoby do sprzeczności.
	W takim razie dla każdej liczby naturalnej w zakresie $1 \dots k$ mamy jakiś antyłańcuch
	i wiemy, że te antyłańcuchy w sumie muszą pokrywać cały poset $P$, co kończy dowód.
\end{proof}
