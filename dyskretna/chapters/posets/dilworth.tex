\begin{theorem}[Twierdzenie Dilwortha]
	Długość maksymalnego antyłańcucha w posecie jest równa ilości łańcuchów w najmniejszym pokryciu łańcuchowym
	(tj. rozkładzie posetu na podzbiory będące łańcuchami). Długość tą dla danego posetu nazywamy szerokością i oznaczamy
	przez $\posetwidth(P)$.
\end{theorem}

\begin{proof}
	Robimy indukcję po liczbie elementów posetu; gdy poset $P$ składa się z jednego elementu twierdzenie zachodzi w trywialny sposób. Załóżmy teraz, że mamy poset składający się z $n$ elementów i o najdłuższym antyłańcuchu długości $k$; antyłańcuch ten nazwiemy $A$. Zdefiniujmy teraz zbiory $U$ i $D$, które będziemy określać jako \textit{upset} i \textit{downset}. Do zbioru $U$ należą wszystkie elementy $P$, takie że są większe od jakiegokolwiek elementu z $A$. Do zbioru $D$ należą zaś wszystkie elementy $P$, takie że są mniejsze od jakiegokolwiek elementu z $A$. Bardziej formalnie:
	\begin{equation*}
		U = \{\,x \in P \mid \exists_{y \in A}  y \leq x \} \setminus A
	\end{equation*}
	\begin{equation*}
		D = \{\,x \in P \mid \exists_{y \in A} y \geq x \} \setminus A
	\end{equation*}

	Pierwsza obserwacja którą należy wykonać, to taka że $U \cup D \cup A = P$. Jest to oczywiste; jeśli istniałby jakiś element z $L$ który nie należałby ani do downsetu ani do upsetu ani do antyłańcucha maksymalnego, to z faktu że nie należy ani do downsetu ani do upsetu wynikałoby, że musiałby należeć do antyłańcucha maksymalnego (bo nie jest porównywalny z żadnym elementem z antyłańcucha maksymalnego).

	Druga obserwacja: nie istnieje element, który należy jednocześnie do upsetu i downsetu. Załóżmy nie wprost, że tak jest: mamy jakieś $x, y, z$ takie, że $x \geq y$,  $x \leq z$, i $y,z \in A$. Wówczas otrzymujemy że $y \leq x \leq z$, a więc z tranzytywności $P$ mamy że $y \leq z$, ale $y, z$ są nieporównywalne bo są w jednym antyłańcuchu. Otrzymana sprzeczność dowodzi obserwację.

	Teraz musimy rozpatrzyć trzy przypadki:
	\begin{enumerate}
		\item $U = D = \emptyset$ \\
		      Bardzo fajny przypadek, głównie dlatego że trywialny do udowodnienia; każdy element z antyłańcucha tworzy jednoelementowy łańcuch zawierający tylko siebie samego, mamy podział $P$ na $k$ łańcuchów.
		\item $U = \emptyset \hspace{5pt} \mathbf{ALBO} \hspace{5pt} D = \emptyset$ \\
		      Weźmy z \(P\) takie \(x, y\), że \(x\) jest elementem maksymalnym, a \(y\) jest elementem minimalnym. Wtedy z tego, że upset lub downset jest pusty, to któryś z nich musi należeć do antyłańcucha. Rozważmy więc poset \(P' = P\setminus \{x,y \}\)\footnote{Formalnie ten zapis nie ma sensu, bo poset nie jest jednoznaczny ze zbiorem na którym jest zdefiniowany, ale chyba wiadomo o co chodzi}. Z założenia indukcyjnego wiemy, że możemy go podzielić na \(k - 1 \) łańcuchów. Tak więc dodając do \(P'\) łańcuch \( \{x, y\} \) otrzymujemy podział \(P\).
		\item $U \not = \emptyset \wedge D \not = \emptyset$ \\
		      Rozpatruję sobie posety na zbiorach $B = A \cup U, C = A \cup D$. Oba z nich z założenia indukcyjnego da się podzielić na $k$ łańcuchów. Ponadto każdy element $A$ ma łańcuch różny od wszystkich innych elementów $A$ w swoim pokryciu łańcuchowym dla zbiorów $B$ i $C$ (2 elementy z jednego antyłańcucha nie mogą być w jednym łańcuchu). W takim razie po prostu ,,sklejam'' łańcuchy z $B$ i $C$ w danym elemencie $A$ i mam poprawne pokrycie łańcuchowe całego posetu $P$.
	\end{enumerate}

\end{proof} % TODO: Dodać ilustrację bijekcji
\begin{proof}[Alternatywny dowód]
	Ten dowód opiera się na twierdzeniu K\H{o}niga. Niech \((P, \preceq)\) będzie posetem. Łatwo zauważyć, że rozmiar największego antyłańcucha jest
	większy lub równy od najmniejszego rozkładu $P$ na łańcuchy -- elementy tego największego antyłańcucha muszą znaleźć się w parami różnych łańcuchach.
	Wystarczy więc udowodnić nierówność w drugą stronę -- że jeżeli mamy \textit{najmniejsze} pokrycie łańcuchowe, to rozmiar największego antyłańcucha jest
	większy lub równy od mocy tego pokrycia.
	Skonstruujmy graf dwudzielny $G = (P, P', E)$, gdzie $P'$ to kopia elementów $P$ (innymi słowy chcemy aby $V = P \sqcup P$, gdzie $V$ to zbiór wszystkich wierzchołków grafu),
	a para $(u, v') \in E$ wtedy, i tylko wtedy, gdy $u \prec v$. Przez $M$ oznaczymy pewne skojarzenie tego grafu. Zauważmy następujący fakt --
	istnieje ,,naturalna'' bijekcja pomiędzy podziałami $P$ na łańcuchy, a skojarzeniami w $G$. Mając podział $R$ na łańcuchy możemy zdefiniować skojarzenie
	przez równoważność $(a, b') \in M$ wtw. $a$ jest bezpośrednio przed $b$ w jednym łańcuchu. Odwracając ten proces, mając pewne skojarzenie
	możemy uzyskać rozkład na łańcuchy ustalając, zaczynając podziału na jednoelementowe łańcuchy, a następnie dla każdego $(a, b') \in M$
	łącząc łańcuchy $(\ldots, a)$ i $(b, \ldots)$ w $(\ldots, a, b, \ldots)$ -- łatwo wykazać, że zawsze $a, b$ będą na końcach, tj. ta operacja jest poprawnie zdefiniowana.
	Co ciekawsze, z konstrukcji bijekcji łatwo zauważyć, że zachodzi równość $\card{R} = \card{P} - \card{M}$, bo dla każdej krawędzi skojarzenia ,,łączymy'' dokładnie dwa łańcuchy (zmniejszając moc początkową $\card{P}$ o $1$).
	Niech $R$ będzie najmniejszym pokryciem łańcuchowym $P$, a $M$ odpowiadającym mu skojarzeniem. Weźmy teraz najmniejsze pokrycie wierzchołkowe $C$ grafu $G$, które z tw. K\H{o}niga ma moc $\card{M}$.
	Definiujemy zbiór pomocniczy $D = \set{x \in P : x \in C \text{ lub } x' \in C}$ -- z tej definicji wynika nierówność $\card{D} \leq \card{C}$.
	Ale zbiór $P \setminus D$ jest antyłańcuchem naszego posetu -- ponieważ $C$ było pokryciem wierzchołkowym, każde dwa elementy spoza $D$ muszą
	nie mieć krawędzi między sobą w zdefiniowanym wcześniej grafie dwudzielnym, czyli przechodząc na posety są one parami nieporównywalne.
	Znaleźliśmy więc antyłańcuch o mocy $\card{P} - \card{D} \geq \card{P} - \card{C} = \card{P} - \card{M} = \card{R}$, co kończy dowód.
\end{proof}

