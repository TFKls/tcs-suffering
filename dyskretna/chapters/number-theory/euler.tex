\subsection{Definicja funkcji \texorpdfstring{$\varphi$}{phi}}
\begin{definition}
	Niech $n \in \natural_1$. Przez $\integer_n^*$ oznaczamy zbiór $\set{x \in \integer: 1 \leq x \leq n, x \perp n}$. \\
	Funkcję $\varphi$ (czasami nazywaną funkcją ,,tocjent'' Eulera) definiujemy jako
	$\varphi(n) = \card{\integer_n^*}$, czyli ilość liczb naturalnych mniejszych i względnie pierwszych z $n$.
\end{definition}

\begin{theorem}
	\label{nt:phigrupa}
	Zbiór $\integer_n^*$ z mnożeniem modulo $n$ jest grupą.
\end{theorem}
\begin{proof}
	Dla przejrzystości, mnożenie modulo $n$ oznaczymy przez $\circ$, a mnożenie w pierścieniu liczb całkowitych przez $\cdot$.

	Aksjomat łączności przechodzi bezpośrednio z pierścienia liczb całkowitych (branie reszty z dzielenia przez $n$ nie zmienia tej własności).

	Oczywiście $1$ jest elementem neutralnym mnożenia oraz należy do $\integer_n^*$ dla każdego $n$.

	Grupa jest również zamknięta na mnożenie -- załóżmy nie wprost, że $a,b \in \integer_n^*$, ale $a \circ b \notin \integer_n^*$.
	Oznacza to, że istnieje jakiś dzielnik pierwszy $p$, że $p \mid n$ oraz $p \mid a \circ b$.
	Z definicji mnożenia modulo wiemy, że $a \circ b = a \cdot b - n \cdot k$ dla pewnego $k \in \natural$, czyli
	$a \circ b + n \cdot k = a \cdot b$. Ale ponieważ $p$ dzieli lewą stronę równania, musi również
	dzielić prawą. Z lematu Euklidesa (patrz. fundamentalne twierdzenie arytmetyki) wynika, że
	$p \mid a$ lub $p \mid b$, a ponieważ $p \mid n$ jest to sprzeczne z założeniem $a, b \in \integer_n^*$.

	Pozostało wykazać jedynie istnienie odwrotności. Niech $a \in \integer_n^*$. Ponieważ
	$\gcd(a, n) = 1$, z tożsamości Bezouta istnieją $x, y$ spełniające:
	$$a \cdot x + n \cdot y = 1.$$ Jak przyjrzymy się wystarczająco długo,
	to widzimy, że z definicji mnożenia modulo wynika $x \equiv a^{-1} \pmod n$.
	Oczywiście $x \perp n$ bo dla dowolnej liczby $d$ dzielącej $x$ oraz $n$
	zachodzi $d \mid 1 \implies d = 1$ (możemy wyjąć $d$ przed nawias w tożsamości).
\end{proof}

\subsection{Twierdzenie Eulera}
\begin{theorem}
	\label{nt:phieuler}
	Dla dowolnego $a \perp n$ zachodzi $a^{\varphi(n)} \equiv 1 \pmod{n}$.
\end{theorem}
\begin{proof}
	Niech $a = n \cdot k + b$, i $b \in \set{0, 1, \ldots, n-1}$. Oczywiście
	zachodzi $a^{\varphi(n)} \equiv b^{\varphi(n)} \pmod(n)$ oraz $a \perp n \implies b \perp n$.
	W takim razie zachodzi $b \in \integer_n^*$. Niech $r$ będzie rzędem $b$ w tej grupie.
	Z twierdzenia Lagrange'a wynika, że $r \mid \varphi(n)$, ponieważ $\varphi(n)$ to z definicji
	$\card{\integer_n^*}$, czyli $b^{\varphi(n)} \equiv (b^r)^{d} \equiv 1^d \equiv 1 \pmod{n}$, co kończy dowód.
\end{proof}

\subsection{Multiplikatywność funkcji \texorpdfstring{$\varphi$}{phi}}
\begin{definition}[Funkcje multiplikatywe]
	Funkcję $f: \natural_1 \to \real$ nazywamy \textit{multiplikatywną}, jeżeli $f(1) = 1$ i dla dowolnych $n, m \in \natural_1$ zachodzi:
	$$n \perp m \implies f(n \cdot m) = f(n) \cdot f(m).$$
\end{definition}

\begin{theorem}
	\label{nt:phimulti}
	$\varphi$ jest funkcją multiplikatywną.
\end{theorem}
\begin{proof}
	Oczywiście $\varphi(1) = \card{\set{1}} = 1$. Niech $n, m \in \natural_1$ i $n \perp m$.
	Z definicji $\varphi$ wiemy, że aby pokazać oczekiwaną równość wystarczy wykazać bijekcję z $\integer_{nm}^*$ w $\integer_n^* \times \integer_m^*$.
	Postawimy hipotezę, że funkcja $f: \integer_{nm}^* \ni x \into (x \bmod n, x \bmod m) \in \integer_n^* \times \integer_m^*$ jest dobrze zdefiniowaną
	bijekcją.
	Dobra definicja wynika prosto z własności dzielenia -- jeżeli $k \mid n$ to $k \mid nm$, czyli $\gcd(a,n) \neq 1 \implies \gcd(a, nm) \neq 1$.
	Z kontrapozycji dostajemy, że $\gcd(a, nm) = 1 \implies \gcd(a, n) = 1$, co wystarcza aby dowieść dobrze zdefiniowaną operację.
	Natomiast injektywność i surjektywność wynika bezpośrednio z chińskiego twierdzenia o resztach.
	Jeżeli istniałoby $x \neq y$ spełniające $(x \bmod n, x \bmod m) = (y \bmod n, y \bmod m)$, to otrzymujemy sprzeczność z
	(istniałyby dwa rozwiązania chińskiego twierdzenia o resztach w przedziale $[nm]_0$). Analogicznie, jeżeli mamy parę $(a, b)$ to rozwiązując układ kongruencji znajdziemy
	odpowiedni element $x \in [nm]_0)$ spełniający $(a, b) = (x \bmod n, x \bmod m)$ oraz $x \perp nm$.\footnote{Jeżeli istniałby nietrywialny wspólny dzielnik $x, nm$ to z lematu Euklidesa musiałby dzielić $n$ lub $m$, co byłoby sprzeczne z wyborem $a, b$}
\end{proof}
Więcej o funkcjach multiplikatywnych opowiemy w sekcji \textit{Splot Dirichleta}.

\subsection{Wzór ,,jawny'' na funkcję \texorpdfstring{$\varphi$}{phi}}
\begin{theorem}
	\label{nt:closedmulti}
	Niech $f$ będzie funkcją arytmetyczną, a $n \in \natural_1$.
	Niech $n = p_1^{\alpha_1}p_2^{\alpha_2}\ldots p_k^{\alpha_k}$, gdzie
	$p_i \in \mathbb P$ oraz $i \neq j \implies p_i \neq p_j$ (wiemy, że taki rozkład zawsze istnieje z fundamentalnego twierdzenia arytmetyki).
	$f$ jest multiplikatywna wtedy i tylko wtedy, gdy zachodzi:
	\begin{equation*}
		f(n) = \prod_{i=1}^{k} f(p_i^{\alpha_i}).
	\end{equation*}
\end{theorem}
\begin{proof}
	Obie implikacje są w miarę łatwe do zauważenia, poniżej udowodnimy mniej trywialną z nich.\footnote{Drugą implikację w sposób klasyczny dla podręczników matematycznych pozostawiamy czytelnikowi.}
	Załóżmy, że $f$ jest multiplikatywne. Wykonamy indukcję po $k$. Dla $k = 0$ zachodzi $n = 1 \implies f(1) = \prod_{k=1}^{0} = 1$.
	Inaczej niech $P = \prod_{i=1}^{k-1} p_i^{\alpha_i}$ -- oczywiście $P \perp p_k^{\alpha_k}$.
	Z indukcji wiemy, że $f(P) = \prod{i=1}^{k-1} f(p_i^{\alpha_i})$, a z multiplikatywności
	otrzymujemy $f(n) = f(P) \cdot f(p_k^{\alpha_k})$, co dowodzi oczekiwanej równości.
\end{proof}

\begin{corollary}
	Niech $n = p_1^{\alpha_1}p_2^{\alpha_2}\ldots p_k^{\alpha_k}$, gdzie $p_i \in \mathbb P$ oraz $i \neq j \implies p_i \neq p_j$.
	Zachodzi:
	\begin{equation*}
		\varphi(n) = \prod_{i=1}^{k} p_i^{\alpha_i-1} \cdot (p_i-1) = n \prod_{i=1}^{k} (1-\frac{1}{p})
	\end{equation*}
\end{corollary}
\begin{proof}
	Oczywiście $\varphi(1) = 1 = \prod_{i=1}^{0} \dots$.
	Z poprzedniego twierdzenia wiemy, że wystarczy udowodnić równość $\varphi(p^{\alpha}) = (p-1) \dot p^{\alpha-1}$ dla każdego $p \in \mathbb P$ -- wtedy twierdzenie otrzymamy bezpośrednio z multiplikatywności $\varphi$.

	Indukujemy się po $\alpha$ -- dla $\alpha = 1$ zachodzi $\varphi(p) = p-1$, ponieważ wszystkie mniejsze liczby naturalne są względnie pierwsze z $p$.
	Niech $\alpha \geq 2$: wtedy $\varphi(p^{\alpha}) = p \cdot \varphi(p^{\alpha-1})$, ponieważ wiemy, że tylko liczby postaci $x + kp^{\alpha-1}$, gdzie $k \in \set{0,1,...,p-1}$ a $x \in \integer_{p^{\alpha-1}}^*$ będą względnie pierwsze z $p$.
	Jest tak ze względu na następujące równoważności:
	$$x \perp p^{\alpha-1} \iff p \mid x \iff p \mid \left(x+p^{\alpha-1}\right).$$
	Ale to oznacza, że $\varphi(p^{\alpha}) = p\cdot\varphi(p^{\alpha-1}) \stackrel{ind.}{=} p\cdot p^{\alpha-2}(p-1) = (p-1)p^{\alpha-1}$.
\end{proof}
