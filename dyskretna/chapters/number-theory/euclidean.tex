
\begin{definition}[Notacja]
	Niech $a, b \in \integer$. Mówimy, że:
	\begin{enumerate}
		\item $a$ \textit{jest podzielne przez} $b$ jeżeli istnieje $k \in \integer$
		      spełniające $a \cdot k = b$. Równoważnie mówimy, że $b$ jest dzielnikiem $a$,
		      i oznaczamy tą relację jako $a \mid b$, a jej negację przez $a \\nmid b$.
		\item liczba $c$ jest \textit{wspólnym dzielnikiem} $a, b$ jeżeli $c \mid a$ oraz $c \mid b$.
		      Przez $\gcd(a, b)$ oznaczamy \textit{największy wspólny dzielnik} liczb $a, b$, tj. największe
		      $k \in \natural$ spełniające $k \mid a$ oraz $k \mid b$.
		\item $a$ jest \textit{względnie pierwsze} z $b$ jeżeli $\gcd(a, b) = 1$.
		      Oznaczamy to równoważnie przez $a \perp b$.
	\end{enumerate}
\end{definition}

\begin{fact}
	Niech $a, b, c, d \in \natural$ i $d \mid a, d \mid b$. Zachodzi:
	\begin{enumerate}
		\item $d \mid -a$
		\item $d \mid a+b$
		\item $d \mid a\cdot c$
		\item $d \mid \gcd(a, b)$
		\item $a \mid c \implies d \mid c$
		\item $c \mid \gcd(a, b) \iff c \mid a \land c \mid b$
	\end{enumerate}
\end{fact}

\begin{theorem}[Algorytm Euklidesa]
	\label{nt:euklides}
	Niech $f: \integer \times \integer \to \natural$ będzie funkcją zdefiniowaną rekurencyjnie jako:
	\begin{equation*}
		f(a, b) = \begin{cases}
			a                & \text{gdy } b = 0 \\
			f(b, a \bmod{b}) & \text{wpp.}
		\end{cases}
	\end{equation*}
	Zachodzi $f = \gcd$.\footnote{Na liczby całkowite $\gcd$ rozszerzamy biorąc $\gcd(a, b) = \gcd(\abs{a}, \abs{b})$.}
\end{theorem}
\begin{proof}
	Zauważmy, że $\gcd(a, 0) = a$ ponieważ dla każdego $a$ istnieje $k = 0$ spełniające $a \cdot k = 0$, czyli $0$ jest podzielne przez wszystkie liczby naturalne.
	Dowodzi to przypadkowi bazowemu rekurencji.
	Wiemy, że $a \bmod b = a - k \cdot b$ dla pewnego $k \in \natural$ -- ale
	z poprzednich faktów możemy wywnioskować, że $d$ jest wspólnym dzielnikiem $a, b$
	wtedy i tylko wtedy, gdy jest wspólnym dzielnikiem $b, a \bmod b$, co kończy dowód.
\end{proof}

\begin{theorem}[Tożsamość Bezouta]
	\label{nt:bezout}
	Niech $a, b \in \integer$ i $d = \gcd(a, b)$. Istnieje nieskończenie
	wiele par liczb $x, y \in \integer$ takich, że $a \cdot x + b \cdot y = d$.
	Liczby $x, y$ nazywa się \textit{współczynnikami Bezouta}.
\end{theorem}
\begin{proof}
	Zauważmy, że mając jedno rozwiązanie $x, y$ możemy otrzymać ich nieskończenie wiele --
	wystarczy wziąć zbiór $\set{(x+bk, y-ak) : k \in \integer}$. Pokażemy więc istnienie jednego
	rozwiązania -- co więcej, pokażemy efektywny algorytm jego otrzymywania.

	Zmodyfikujemy nieco w tym celu algorytm Euklidesa -- zamiast zwracać jednej liczby będzie zwracał ich trójkę.
	Wynikiem $f(a, b)$ będzie trójka $(d, s, t)$ spełniająca warunki $d = \gcd(a, b)$ oraz $a \cdot s + b \cdot t = d$.
	Dla przypadku bazowego $b = 0$ łatwo zauważyć, że działa trójka $(a, 1, 0)$, bo $d = a = a \cdot 1 + b \cdot 0$.
	Odrobinę trudniej jest dla przypadku rekurencyjnego. Niech $(d', s', t')$ to wartości wywołania funkcji
	$f(b, a \bmod b)$. Oczywiście $d = d'$, ale ciężej podejść do wartości $s, t$. Aby to zrobić potrzebujemy
	skorzystać z pomocy brata operacji modulo -- dzielenia bez reszty\footnote{Zwykle idą w parze jako dzielenie z resztą, lecz dla uproszczenia nie wprowadziliśmy go przy podstawowym algorytmie Euklidesa}.
	Niech $q = \floor{\frac{a}{b}}$. Z definicji wiemy, że $q \cdot b + a \bmod b = a$. Ale to oznacza, że:
	\begin{align*}
		d & = b \cdot s' + (a \bmod b) \cdot t'       \\
		  & = b \cdot s' + (a - q \cdot b) \cdot t'   \\
		  & = a \cdot t' + b \cdot (s' - q \cdot t'),
	\end{align*}
	co daje nam wzór na $s = t'$ i $t = s - q \cdot t'$.

	Nasza końcowa funkcja wygląda więc następująco:
	\begin{equation*}
		f(a, b) = \begin{cases}
			(a, 1, 0)         & \text{gdy } b = 0
			(d, t, s - t * q) & \text {wpp, gdzie } q = \floor{\frac{a}{b}}, (d, s, t) = f(b, a \bmod b)
		\end{cases}
	\end{equation*}
	Powyższy algorytm obliczania współczynników Bezouta nazywa się \textit{rozszerzonym algorytmem Euklidesa}.
\end{proof}

\begin{corollary}
	Dla dowolnego $a, b \in \natural$, $d = \gcd(a, b)$ zachodzi:
	\begin{equation*}
		\set{a \cdot x + b \cdot y : x, y \in \integer} = \set{d \cdot k : k \in \integer}.
	\end{equation*}
\end{corollary}
