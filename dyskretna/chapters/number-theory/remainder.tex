\begin{theorem}[Chińskie twierdzenie o resztach]
	Niech $n, m \in \natural_1, n \perp m$, $a \in [n]_0, b \in [m]_0$.
	Wtedy istnieje dokładnie jedno $x \in [nm]_0$ spełniające układ kongruencji:
	\begin{equation*}
		\begin{cases}
			x \equiv a \pmod n \\
			x \equiv b \pmod m
		\end{cases}
	\end{equation*}
\end{theorem}

\begin{proof}[Dowód unikalności]
	Załóżmy, że istnieją $x, y \in [nm]_0$ obydwa spełniające układ równań.
	Niech $z = x-y$. Zachodzi:
	\begin{equation*}
		\begin{cases}
			x - y \equiv a - a \equiv 0 \pmod n, \\
			x - y \equiv b - b \equiv 0 \pmod m,
		\end{cases}
	\end{equation*}
	czyli $n \mid z, m \mid z \implies nm \mid z$. Ale to oznacza, że $z = 0$, bo analizując możliwe wartości
	otrzymujemy, że  $z \in \set{-nm+1, \ldots, nm-1}$, a $0$ jest jedyną liczbą podzielną przez $nm$ w tym przedziale.
	Czyli $x-y = 0 \implies x = y$.
\end{proof}
\begin{proof}[Dowód istnienia ,,ala MFI'']
	Niech $f: [nm]_0 \ni x \mapsto (x \bmod n, x \bmod m) \in [n]_0 \times [m]_0$.
	Z dowodu unikalności wiemy, że ta funkcja jest injekcją -- ale
	ponieważ $\card{[nm]_0} = nm = \card{[n]_0 \times [m]_0}$, funkcja $f$ musi być
	również surjektywna (inaczej moc by się nie zgadzała).
\end{proof}
\begin{proof}[Dowód istnienia przez konstrukcję]
	Wiemy, że $n \perp m$, czyli z tożsamości Bezouta istnieją $k, l$
	spełniające $n \cdot k + m \cdot l = 1$.
	Niech $x = m \cdot l \cdot a + n \cdot k \cdot b\pmod{nm}$. Wtedy zachodzi:
	\begin{align*}
		x & = m \cdot l \cdot a + n \cdot k \cdot b & \pmod n  \\
		  & \equiv (1 - n \cdot k) \cdot a + 0      & \pmod n  \\
		  & \equiv 1 \cdot a \equiv a               & \pmod n; \\
		x & = m \cdot l \cdot a + n \cdot k \cdot b & \pmod m  \\
		  & \equiv 0 + (1 - m \cdot l) \cdot b      & \pmod m  \\
		  & \equiv 1 \cdot b \equiv b               & \pmod m,
	\end{align*}
	czyli $x$ jest poprawnym rozwiązaniem układu równań, co kończy dowód.
\end{proof}

\begin{corollary}
	Niech $n_1, n_2, \ldots, n_k \in \natural_k$ będą parami względnie pierwsze,
	a $a_i \in [n_i]$ dla każdego $i$. Wtedy istnieje dokładnie jedno $x \in [\prod_{i=1}^{k} n_i]$
	spełniające układ kongruencji:
	\begin{equation*}
		\begin{cases}
			x \equiv a_1 & \pmod {n_1} \\
			x \equiv a_2 & \pmod {n_2} \\
			\vdots                     \\
			x \equiv a_k & \pmod {n_k}
		\end{cases}
	\end{equation*}
\end{corollary}
\begin{proof}
	Aplikujemy chińskie twierdzenie o resztach $k-1$ razy, najpierw dla $n_1$ i $n_2$,
	potem dla $n_1n_2$ i $n_3$, itd.
\end{proof}
